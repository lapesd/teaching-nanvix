\documentclass[11pt]{article}
\usepackage[utf8]{inputenc}

\newif\ifbr
\newif\ifen

% Output Language
%\brtrue \usepackage[brazilian]{babel}
\entrue \usepackage[english]{babel}

\usepackage[parfill]{parskip}

\ifbr
\title{Substituição de Páginas}
\author{
	Pedro H. Penna, Henrique Freitas,\\%
	Márcio Castro and Jean-François Méhaut\\[0.3em]
	\small Pontifical Catholic University of Minas Gerais\\
	\small Federal University of Santa Catarina\\
	\small University of Grenoble Aples}
\date{}
\else\ifen
\title{Page Replacement}
\author{
	Pedro H. Penna, Henrique Freitas,\\%
	Márcio Castro and Jean-François Méhaut\\[0.3em]
	\small Pontifical Catholic University of Minas Gerais\\
	\small Federal University of Santa Catarina\\
	\small University of Grenoble Aples}
\date{}
\fi\fi

\begin{document}

\maketitle

\begin{abstract}
\ifbr
	\noindent A memória de um computador é organizada em uma hierarquia,
	com o sistema operacional atuando como gerente. Nessa organização,
	uma função fundamental é a de manter páginas utilizadas em memória e
	descarregar as não necessárias em disco. No Nanvix, essa tarefa é de
	responsabilidade do substituidor de páginas e nesse projeto você
	deverá propor melhorias a ele.
\else\ifen
	\noindent The memory of a computer is hierarchically organized, with
	the operating system acting as a manager. In this approach, a core
	functionality is to maintain in memory the most used pages, and
	offload to the disk those that are not used. In nanvix, this job is
	accomplished by the page replacement component, and in this
	assignment you should propose enhancements in it.
\fi\fi
\end{abstract}

\ifbr
	\subsubsection*{Fundamentação Teórica}

		Para proporcionar a ilusão de um computador com memória
		infinitamente grande, rápida e não volátil, a memória de um
		computador é organizada em uma hierarquia, com memórias
		pequenas, voláteis e de acesso rápido nos primeiros níveis, e
		memórias lentas, densas e não voláteis nos últimos.

		Nessa hierarquia, o papel do sistema operacional é o de realizar
		a gerência de maneira eficiente e transparente ao programador.
		Então, para alcançar esse objetivo, sistemas operacionais
		modernos, recorrem a uma técnica híbrida de \textit{hardware} e
		\textit{software}, denominada memória virtual.

		Nessa técnica, o espaço de endereçamento de um processo é
		dividido em regiões de mesmo tamanho, denominadas páginas, que,
		por sua vez, são mapeados em regiões de tamanho correspondente
		na memória física, denominadas quadros de página. Assim, um
		conjunto de processos pode compartilhar a memória física, mesmo
		que não caibam integralmente nela.

		Para tanto, tudo o que o sistema operacional deve fazer é manter
		as páginas mais utilizadas na memória, armazenando em disco
		aquelas que não são mais necessárias. Essa gerência exerce
		influência direta no desempenho final do sistema e é de
		responsabilidade do substituidor de páginas, um componente do
		módulo de gerenciamento de memória.
\else\ifen
	\subsubsection*{Background}

		To enable the illusion of a computer with a infinitely large,
		fast and non-volatile memory, the memory of computer is
		hierarchically organized. In this approach, fast and small memories
		are placed near the processor, whereas large and slow memories
		are farther.

		In the memory hierarchy, the job of the operating system is to
		act of a manager, in a efficient and transparent way. To do so,
		modern operating systems rely on a hybrid hardware/software
		technique called virtual memory.

		In this technique, the address space of a process is divided in
		equal-size regions, named pages. In turn, pages are mapped to
		regions of the same size in the physical memory, which are
		referred as page frames. This way, a set of process can share
		the physical memory, even though all of them may not fit in at
		the same time.

		To maintain such illusion based on pages and page frames, all
		that the operating system does is to maintain in memory those
		pages that are more frequently used, and offload to disk those
		that are no logger. This bookkeeping has a great influence in
		the overall performance of the system, and it is carried out by
		the page replacement component, a module of the memory
		management subsystem.
\fi\fi
\ifbr
	\subsubsection*{Descrição do Projeto}

		O componente de substituição de páginas do Nanvix adota um
		política local \textit{first-in-first-out}. Isto é, quando uma
		falta de página ocorre, o sistema operacional escolhe, dentro
		das páginas residentes em memória do processo que falhou, a
		página mais antiga para ser substituída. Essa política tem como
		principais vantagens a simplicidade na implementação e
		\textit{overhead} mínimo na operação de substituição em si.

		No entanto, essa política impõe sérios problemas de desempenho
		ao sistema, principalmente se processos possuírem uma padrão de
		acesso aos dados -- o que geralmente é verdade. Por exemplo,
		suponha um programa que efetue a multiplicação de duas matrizes
		grandes, que não cabem na memória. Nesse caso, manter em memória
		as paginas novas certamente não é a escolha mais sábia, mas sim
		manter as páginas que representem o conjunto de trabalho do
		processo, que são as linha e colunas das matrizes operando e
		partes da matriz resultado.

		Para resolver esse problema, você deverá modificar o algoritmo
		de substituição de páginas do Nanvix e implementar qualquer
		outra política que conduza a melhores desempenhos. Você deverá
		avaliar o desempenho da sua solução quantitativamente. Para
		isso, você pode utilizar um código exemplo que efetua a
		multiplicação de matrizes.  Tenha em mente que algoritmos
		robustos, como conjunto de trabalho e \textit{aging}, podem
		conduzir a bons resultados.
\else\ifen
	\subsubsection*{Assignment Description}

		The page replacement component in Nanvix relies on a
		first-in-first-out local page replacement policy. That is,
		whenever a page fault occurs, the operating system chooses,
		among the in-core pages of the faulting process, the oldest page
		to be evicted. This policy has is easy to implement and yields
		to minimum overhead. However, it leads to a poor performance,
		specially those process that present a well-behaved data access
		pattern -- which is usually true.

		For instance, suppose program that performs a matrix multiplication of two
		matrices that do not fit in memory. In this situation, the
		wisest choice is not to keep in memory the most recently used
		pages, but to maintain in-core those that represent the working
		set of the program, which are the pages that store current lines
		and columns that are being multiplied.

		To address this problem, you have to hack the page replacement
		algorithm in Nanvix and implement any other policy that yields
		to a better performance. You have to present a quantitative
		performance analysis of your solution. For doing so, you may use
		the matrix multiplication program that we have provided along
		with the system. Keep in mind that robust algorithms like the
		Working Set and Aging may yield to good results.

\fi\fi

\ifbr
	\subsubsection*{Por Onde Começar?}

		O código do subsistema de memória do Nanvix está no diretório
		\texttt{kernel/mm}, dividido em vários arquivos:
		%
		\begin{itemize}
			\item \texttt{mm.c}: inicialização do gerenciador de memória.
			\item \texttt{paging.c}: sistema de paginação e \textit{swapping}.
			\item \texttt{region.c}: gerenciamento de regiões.
		\end{itemize}

	Nesse projeto, você deve se atentar ao arquivo \texttt{paging.c},
	principalmente na função \texttt{allocf()}, que implementa a função
	de substituição de páginas. Além disso, desses arquivos, você também
	usar o exemplo de multiplicação de matrizes em
	\texttt{src/sbin/test/test.c}.

	\subsubsection*{Indo Além}

		Caso você queria implementar uma política de substituição de
		páginas mais robusta, é interessante você estudar a
		\texttt{pfault()}, que lida com faltas de página. Uma ideia é
		modificá-la para manter o registro de faltas de página e então,
		tomar decisões mais interessantes.

		Você será recompensado pelo seu esforço. Procure o professor da
		disciplina para discutir o assunto em maiores detalhes. Além
		disso, o grupo que possuir a implementação mais robusta será
		convidado a submeter um \textit{pull} no repositório de
		desenvolvimento do Nanvix.

\else\ifen
	\subsubsection*{Start Point}

		The implementation of the memory management subsystem is in the
		\texttt{kernel/mm} directory, and it is split in several files:
		%
		\begin{itemize}
			\item \texttt{mm.c}: inicialização do gerenciador de memória.
			\item \texttt{paging.c}: sistema de paginação e \textit{swapping}.
			\item \texttt{region.c}: gerenciamento de regiões.
		\end{itemize}
		
		In this assignment, you should focus on the \texttt{paging.c}
		source file. More precisely, in the \texttt{allocf()} kernel
		function, which implements the paging replacement algorithm.
		Besides this, the source file for the matrix multiplication
		algorithm is located in \texttt{src/sbin/test/test.c}
\fi\fi

\end{document}
