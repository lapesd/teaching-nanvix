\documentclass[11pt]{article}
\usepackage[utf8]{inputenc}

\newif\ifbr
\newif\ifen

% Output Language
%\brtrue \usepackage[brazilian]{babel}
\entrue \usepackage[english]{babel}

\usepackage[parfill]{parskip}

\ifbr
\title{Prefetching}
\author{
	Pedro H. Penna, Henrique Freitas,\\%
	Márcio Castro and Jean-François Méhaut\\[0.3em]
	\small Pontifical Catholic University of Minas Gerais\\
	\small Federal University of Santa Catarina\\
	\small University of Grenoble Aples}
\date{}
\else\ifen
\title{Prefetching}
\author{
	Pedro H. Penna, Henrique Freitas,\\%
	Márcio Castro and Jean-François Méhaut\\[0.3em]
	\small Pontifical Catholic University of Minas Gerais\\
	\small Federal University of Santa Catarina\\
	\small University of Grenoble Aples}
\date{}
\fi\fi

\begin{document}

\maketitle

\begin{abstract}

\ifbr
	\noindent O subsistema de E/S deve prover ao usuário acesso
	eficiente aos dispositivos de E/S. Se isso não ocorrer, o usuário
	certamente ficará frustrado com a sua experiência virtual. Nesse
	projeto você implementará uma técnica que significativamente
	potencializa o desempenho de dispositivos de armazenamento em massa,
	o \textit{prefetching}.
\else\ifen
	\noindent The I/O subsystem should provide to the user efficient
	access to I/O devices. If it does not, the user will certainly
	become frustrated with his/her virtual experience. In this
	assignment you should implement a powerful technique that boosts the
	performance of storage devices: disk block prefetching.
\fi\fi


\end{abstract}

\ifbr
	\subsubsection*{Fundamentação Teórica}

		Dentre os diferentes dispositivos de E/S existentes, podemos
		citar discos magnéticos como fundamentais, por servirem como
		principal local para persistência de dados de usuários e ao
		próprio sistema operacional como meio para possibilitar a
		técnica de memória virtual.

		No entanto, esses dispositivos possuem o inconveniente de
		apresentarem baixas taxas de leitura e escrita, tornando
		operações lentas. Por exemplo, uma única requisição de leitura
		pode levar até 30 ms, em contraste a 0.1 ms para a mesma
		operação em memória. O principal motivo disso, se deve ao fato
		de discos magnéticos serem dispositivos escensialmente
		mecânicos.

		Por isso, para minimizar os efeitos negativos de desempenho,
		técnicas, tais como agendamento de requisições, \textit{caching}
		e \textit{prefetching}, são usualmente aplicadas. Na técnica de
		agendamento, requisições de disco são agendadas de forma a
		minimizar o tempo de movimentação do braço mecânico do disco
		magnético. Na técnica de \textit{caching}, blocos mais usados do
		disco são mantidos em uma \textit{cache} em memória, de forma
		que requisições podem ser satisfeitas mais prontamente. Por fim,
		na técnica de \textit{prefetching}, dados são lidos
		antecipadamente do disco para a \textit{cache} do disco em
		memória, assim requisições de leitura podem também serem
		satisfeitas mais rapidamente.

\else\ifen
	\subsubsection*{Background}

		Among the several existing I/O devices, one can point out
		storage devices as one of the most important. They provide a
		place in which the user can persist some data and the operating
		system may use to implement virtual memory.

		However, these devices have the drawback of presenting slow
		read/write ratios, thereby yielding to poor performance. For
		instance, a single read request may take up to 30 ms, in
		contrast to 1 ms for the same request in memory. The rationale
		behind this comes from the fact that storage devices are
		essentially mechanical devices.

		For this reason, and to address performance issues, techniques
		such as request scheduling, disk block caching and disk block
		prefetching are frequently employed. In the first technique,
		read/write requests are scheduled so as to minimize the
		repositioning time of the disk arm/head. In the second
		technique, most used blocks are kept in a in-core memory, so
		that requests on these blocks can be served faster. Finally, in
		the third technique data is read in advance to the in-core disk
		block buffer, thereby speeding up linear read requests.

\fi\fi

\ifbr
	\subsubsection*{Descrição do Projeto}

		O subsistema de E/S do Nanvix adota a técnica de
		\textit{caching} para maximizar seu desempenho. Requisições
		entre processos de usuário e dispositivos de saída são atendidas
		diretamente de \textit{buffers} de blocos, em memória. O
		gerenciador de E/S mantém \textit{buffers} em uma \textit{cache}
		e utiliza um esquema LRU (\textit{least recently used}) para
		fazer a substituição de \textit{buffers} antigos por
		\textit{buffers} novos, quando necessário.

		Apesar de simples, essa técnica confere um bom desempenho ao
		subsistema de E/S, tanto em requisições de leitura quanto de
		escrita. Para requisições de escrita especificamente, o uso de
		\textit{buffers} possibilita \textit{dealyed write}, reduzindo a
		banda média de transferência -- \textit{buffers} serão
		utilizados ao máximo antes de serem escritos de volta em disco.
		Para requisições de leitura e escrita, o uso da \textit{cache}
		possibilita que requisições sejam atendidas mais prontamente.

		Ainda assim, o desempenho do subsistema de E/S do Nanvix ser
		melhorado ainda mais. Por exemplo, imagine um processo que
		realiza acesso sequencial a um arquivo, como um mp3
		\textit{player}. Quando apenas a técnica de \textit{caching} é
		usada, toda vez que uma requisição de leitura de um novo bloco
		for feita, uma falta na \textit{cache} será gerada e o bloco
		deverá ser carregado diretamente do disco. Alternativamente,
		a técnica de \textit{prefetching} pode ser adotada para que
		blocos seguintes sejam carregados antecipadamente,
		aumentando a taxa de acertos na \textit{cache}. Assim, seu
		trabalho se resume a implementar essa funcionalidade no
		Nanvix.

\else\ifen
	\subsubsection*{Assignment Description}

		The I/O subsystem in Nanvix uses the caching technique to
		speedup disk-use performance. Requests of user processes are
		directly served from in-memory buffers which the kernels
		maintains in a LRU (least recently used) in-core cache.

		Although simple, this technique yields to a better overall
		performance for the I/O subsystem in both, read and write
		requests. More precisely, caching enables delayed write for
		write requests and data reuse for read requests, thus maximizing
		the average disk-bandwidth.
		
		Nevertheless, the overall performance of the I/O subsystem may
		be further improved with prefetching. For instance, suppose a
		scenario in which a process performs linear access to a file,
		like a mp3 does. When only the caching technique is employed,
		whenever a read request is issued, a cache miss will raise and
		the request will have to be served from the disk. Alternatively,
		with prefetching, next blocks in a linear read are loaded in
		advance asynchronously, and thus increase cache hits. Therefore,
		in this assignment you should implement the prefetching
		technique in Nanvix.	

\fi\fi

\ifbr
	\subsubsection*{Por Onde Começar?}

		O código do gerenciador de E/S do Nanvix está no diretório
		\texttt{kernel/fs}, dividido em vários arquivos. No entanto,
		nesse projeto, você deve se atentar ao arquivo
		\texttt{buffer.c}, onde todas as funções para o gerenciamento da
		\textit{cache} de \textit{buffers} estão implementados. Em
		especial, estude a funação \texttt{bread()}, que realiza a
		leitura de um \textit{buffer} de bloco. Uma boa ideia seria
		utilizar essa função para criar a nova funcionalidade de
		\textit{prefetching}. Para testar sua estratégia de
		\textit{prefetching} use o utilitario \texttt{test}.

	\subsubsection*{Recompensas}

		O grupo que possuir a implementação mais robusta será convidado
		a submeter um \textit{pull} no repositório de desenvolvimento do
		Nanvix.

\else\ifen

	\subsubsection*{Start Point}

		The implementation of the I/O subsystem is in the
		\texttt{kernel/fs} directory, and it is split in several files.
		However in this assignment, you should focus on the
		\texttt{buffer.c} file, where all the functions and routines for
		disk-block buffer cache are implemented. More precisely, you
		should study the \texttt{bread()} function, which serves read
		requests from the cache. A good idea is to use rely on the
		implementation of this function to implement a \texttt{breada()}
		function which performs normal read followed by prefetched
		reads. You should hack the \texttt{read()} system call to make
		use of this new feature, and you should rely on the
		\texttt{test} utility to benchmark your solution.

\fi\fi

\end{document}
